\chapter{Giới thiệu}

\section{Đặt vấn đề}

Trong bối cảnh hiện nay, tốc độ chuyển đổi số đang diễn ra nhanh chóng và không ngừng gia tăng. Các thiết bị, vật dụng hằng ngày không ngừng được cải tiến, nâng cấp nhằm nâng cao trải nghiệm người dùng. Những công nghệ như đồng hồ thông minh, đèn điện tự động, khóa cửa từ, cảnh báo thông minh, nhiệt kế thông minh, trợ lý ảo đang dần trở thành một phần tất yếu trong không gian sống của mỗi cá nhân.

Mạng lưới kết nối các thiết bị thông minh và công nghệ tạo điều kiện thuận lợi cho hoạt động giao tiếp giữa thiết bị với đám mây, cũng như giữa các thiết bị với nhau, tất cả đã hình thành nên một mạng lưới mang tên Internet of Things (\gls{iot}), hay còn gọi là Internet vạn vật. Đây là nền tảng đóng vai trò quan trọng trong việc kiến tạo nên môi trường sống thông minh, tiện nghi và linh hoạt. Các thiết bị này ngày nay không còn chỉ thực hiện các chức năng đơn lẻ. Thông qua khả năng thu thập dữ liệu từ các cảm biến về môi trường, chúng đã có thể hỗ trợ cải thiện hệ thống một cách chủ động, hệ quả là nâng cao chất lượng trải nghiệm người dùng.

% tiếp

Tuy nhiên, khả năng thu thập dữ liệu này cũng có mặt hạn chế nhất định. Trong đó, có thể kể đến chính là lượng dữ liệu được gửi liên tục về hệ thống theo thời gian thực, đòi hỏi phải được xử lý với tốc độ nhanh và độ chính xác cao. Đây thực sự là bài toán khó đối với các hệ thống xử lý dữ liệu từ khâu thiết kế, triển khai cho đến vận hành.

Nhiều hệ thống xử lý luồng dữ liệu thời gian thực, điển hình là Apache Storm vốn không hề được trang bị cơ chế tài nguyên linh hoạt. Điều này khiến cho việc quản lý và duy trì vận hành hệ thống một cách thủ công trở nên tốn kém, thiếu hiệu quả, và dễ xảy ra sai sót.

Từ đây, vấn đề được đặt ra là: Làm sao để hệ thống có khả năng co/dãn linh hoạt, chủ động - vừa tối ưu tài nguyên, vừa đảm bảo hiệu năng ổn định - không chỉ ở thời điểm hiện tại mà còn thích ứng tốt trong tương lai khi cả quy mô hệ thống lẫn khối lượng dữ liệu đều sẽ tăng theo thời gian? Trên cơ sở đó, nghiên cứu này đề xuất giải pháp về co dãn tài nguyên đa cấp độ theo hướng chủ động, đồng thời vẫn tận dụng các ưu điểm vượt trội về khả năng xử lý thời gian thực, khả năng chịu lỗi và đảm bảo dữ liệu của hệ thống SmartHome đã được triển khai trên nền tảng Apache Storm.

\section{Mô tả bài toán}

Trong các hệ thống nhà thông minh thời gian thực, khả năng xử lý dữ liệu một cách nhanh chóng và chính xác là yếu tố then chốt để đảm bảo tính liên tục trong tương tác giữa người dùng và thiết bị. Khi số lượng thiết bị và cảm biến không ngừng gia tăng, lượng dữ liệu sinh ra cũng trở nên dày đặc và phức tạp hơn, đặc biệt là khi dữ liệu được phát sinh dưới dạng dòng liên tục (streaming data).

Việc xử lý hiệu quả các dòng dữ liệu này đòi hỏi một nền tảng tính toán có tính phân tán, linh hoạt và khả năng mở rộng cao, đủ năng lực phản hồi kịp thời trước những biến đổi không ngừng của tải hệ thống. Tuy nhiên, phần lớn các nền tảng xử lý truyền thống thiếu cơ chế thích ứng tự động với sự biến động của tải, khiến hệ thống dễ rơi vào tình trạng quá tải khi dữ liệu tăng đột biến hoặc lãng phí tài nguyên khi lưu lượng thấp.

Trái lại, môi trường điện toán đám mây công cộng cung cấp một nền hạ tầng linh hoạt với khả năng phân bổ tài nguyên theo yêu cầu sử dụng thực tế – một trong những đặc điểm cốt lõi của kiến trúc điện toán hiện đại. Chính nhờ đặc tính này, việc kết hợp giữa các nền tảng xử lý dữ liệu theo thời gian thực và hạ tầng đám mây đã mở ra cơ hội xây dựng hệ thống thích nghi linh hoạt với tải dữ liệu, mở rộng hoặc thu hẹp tài nguyên xử lý một cách chủ động.

Trên tinh thần đó, đề tài hướng đến việc phát triển một giải pháp kỹ thuật tích hợp cơ chế co/dãn tài nguyên chủ động vào hệ thống xử lý dữ liệu dòng thời gian thực, với sự hỗ trợ của thuật toán học tăng cường (Reinforcement Learning), cụ thể là Q-learning. Mục tiêu của giải pháp là tối ưu hóa hiệu suất và độ thích nghi của hệ thống SmartHome, đảm bảo khả năng vận hành ổn định trong môi trường điện toán đám mây năng động và biến đổi liên tục.

% Hệ thống SmartHome trong thời gian thực yêu cầu khả năng xử lý dữ liệu nhanh chóng và chính xác để đáp ứng yêu cầu tương tác liên tục giữa người dùng và thiết bị. Với sự gia tăng đáng kể số lượng thiết bị và lượng dữ liệu cảm biến, yêu cầu xử lý trở nên phức tạp, đặc biệt là khi dữ liệu phát sinh theo dạng luồng liên tục (streaming data).

% Để xử lý hiệu quả các dòng dữ liệu này, hệ thống cần sử dụng nền tảng xử lý dữ liệu luồng mạnh mẽ, có khả năng phân tán và mở rộng linh hoạt. Apache Storm là một nền tảng mã nguồn mở nổi bật trong việc xử lý luồng dữ liệu theo thời gian thực. Tuy nhiên, Storm không được thiết kế để hỗ trợ cơ chế co dãn tài nguyên tự động, khiến hệ thống dễ bị quá tải khi lượng dữ liệu tăng cao hoặc gây lãng phí tài nguyên khi dữ liệu thấp.

% Trong khi đó, môi trường điện toán đám mây công cộng cung cấp khả năng mở rộng tài nguyên linh hoạt theo nhu cầu - một trong những đặc tính cốt lõi của điện toán đám mây hiện đại. Chính đặc tính này đã tạo nên một sự kết hợp hoàn hảo giữa Apache Storm với hạ tầng tại môi trường điện toán đám mây công cộng, thông qua môi trường này, ta có thể bổ sung cho Apache Storm một khả năng mà bản thân nó không có - có dãn tài nguyên theo tải dữ liệu.

% Đề tài đặt ra mục tiêu xây dựng một giải pháp kỹ thuật kết hợp Apache Storm với cơ chế co dãn tài nguyên chủ động, được điều phối bởi một thuật toán học tăng cường (Reinforcement Learning), cụ thể là thuật toán Q-learning. Giải pháp này hướng tới việc tối ưu hóa hiệu suất hệ thống xử lý dữ liệu SmartHome theo thời gian thực trên nền tảng điện toán đám mây.
% Đặc biệt, trong giải pháp đề xuất, các máy ảo trên đám mây được tổ chức và quản lý dưới dạng một cụm Docker (Docker cluster), bao gồm nhiều node. Tại đây, mỗi node sẽ tương ứng với một hoặc nhiều supervisor trong hệ thống Storm và toàn bộ cụm có thể được tùy biến điều chỉnh thông qua docker API rất thuận tiện trong việc quản lý. Việc sử dụng Docker giúp tạo ra một kiến trúc thống nhất, có thể hoạt động trong nhiều môi trường bất kể là điện toán đám mây hay hệ thống máy chủ có sẵn. Không những vậy, Docker còn tạo điều kiện thuận lợi cho việc tự động hóa co dãn tài nguyên ở cả mức hệ thống và dịch vụ.

% Thông qua giải pháp học tăng cường, tác giả muốn giải quết bài toán được đặt ra là: \textit{Làm thế nào để tích hợp khả năng co dãn chủ động - dựa trên học tăng cường Q-learning - vào hệ thống xử lý dữ liệu thời gian thực Apache Storm nhằm khai thác tối đa tính mở rộng linh hoạt của public cloud?}

% Giải quyết bài toán này không chỉ giúp tối ưu vận hành cho ứng dụng SmartHome trong môi trường thực tiễn, mà còn nghiên cứu về xu hướng phát triển nổi bật của các hệ thống xử lý dữ liệu \gls{iot} hiện nay khi tích hợp các nền tảng xử lý dữ liệu phân tán với khả năng mở rộng linh hoạt của public cloud.

\section{Đóng góp của đồ án}

Đồ án phát triển dựa trên hệ thống xử lý dữ liệu Storm SmartHome được xây dựng trên nền tảng Apache Storm. Các thành phần hệ thống như mô phỏng dữ liệu nhà thông minh, thiết lập cụm Apache Storm và topology SmartHome để xử lý dữ liệu đã được hiện thực từ nghiên cứu trước của tác giả Đoàn Nho Lâm \autocite{fimocodestormsmarthome}. Đồ án hiện tại kế thừa hệ thống đó và tiếp tục phát triển, bổ sung vào cụm Apache Storm ba thành phần chính:
% Hệ thống trong đồ án được xây dựng dựa trên hệ thống xử lý dữ liệu lớn Storm SmartHome được tác giả Đoàn Nho Lâm xây dựng dựa trên nền tảng Apache Storm \autocite{fimocodestormsmarthome}. Trong hệ thống đã có các thành phần \gls{mqtt} publisher, \gls{mqtt} broker, cụm Apache Storm. Phần tiếp theo của đồ án sẽ trình bày các đóng góp của tác giả vào hệ thống có sẵn.

\subsection{Storm exporter}

Để phục vụ mục tiêu giám sát và đánh giá hiệu suất của hệ thống co/dãn tài nguyên, tác giả đã phát triển một công cụ mang tên Storm Exporter, dựa trên phiên bản trước do tác giả Đỗ Anh Tú phát triển \autocite{mr4x2stormexporterprometheus}. Chương trình được viết lại bằng ngôn ngữ lập trình Go, đồng thời tích hợp Docker API để thu thập các chỉ số liên quan đến container.

Dữ liệu được thu thập định kỳ, với chu kỳ lấy mẫu được cấu hình linh hoạt thông qua biến môi trường. Storm Exporter đóng vai trò như một thành phần trung gian, chịu trách nhiệm thu thập và chuyển hóa các chỉ số hoạt động (metrics) từ cụm Apache Storm, cụm supervisor chạy trên Docker Compose hoặc trên môi trường điện toán đám mây đưa chúng về định dạng chuẩn để tích hợp với hệ thống giám sát như Prometheus. Ngoài ra, các chỉ số thu thập được còn đóng vai trò đầu vào cho hệ thống dự đoán dữ liệu - Storm Forecast nhằm hỗ trợ việc ra quyết định co/dãn tài nguyên một cách chính xác và kịp thời.

\subsection{Storm Forecast}

Storm Forecast là thành phần trung tâm và chủ động trong hệ thống đề xuất. Thông qua dữ liệu đầu vào từ Storm Exporter, chương trình chủ động dự đoán nhu cầu tài nguyên của các supervisor trong tương lai, từ đó tự động đưa ra các quyết định co/dãn thích ứng. Sau mỗi hành động, Storm Forecast quan sát phản hồi từ môi trường, liên tục điều chỉnh chiến lược vận hành và tự học hỏi nhằm tối ưu hiệu suất hệ thống theo thời gian. Storm Forecast được hiện thực bằng ngôn ngữ lập trình Python, sử dụng thư viện Gymnasium \autocite{gymnasium} giúp cung cấp một \gls{api} chuẩn hóa cho các bài toán học tăng cường.

\subsection{Storm Autoscaler}

Thành phần này chịu trách nhiệm thực thi các hành động co/dãn tài nguyên đa cấp độ, bao gồm việc điều chỉnh số lượng supervisor và thay đổi phân bố các tiến trình spout và bolt trên các supervisor, dựa trên chỉ thị từ Storm Forecast. Chương trình được hiện thực bằng ngôn ngữ Go, cho phép co/dãn tài nguyên các supervisor trong môi trường Docker Compose hoặc môi trường \gls{gcp}.

\subsection{Đóng gói chương trình thành các image}

Chương trình Storm Exporter và Storm Forecast đã được cung cấp tệp chỉ dẫn Dockerfile nhằm xây dựng các image phục vụ triển khai trong môi trường đa-container sử dụng Docker Compose. Việc container hóa hai thành phần này cho phép tận dụng các lợi ích nổi bật của Docker, bao gồm khả năng mở rộng nhanh chóng, phân phối mã nguồn thuận tiện, tốc độ triển khai cao, và tính nhất quán khi vận hành trên nhiều nền tảng máy chủ khác nhau.

Ngược lại, chương trình Storm Autoscaler – do đặc thù cần tương tác trực tiếp với Docker Compose, hiện vẫn chưa có giải pháp \gls{api} phù hợp – nên được triển khai dưới dạng một chương trình chạy trực tiếp trên máy chủ điều khiển chạy tiến trình Storm nimbus trong cụm Apache Storm.

\section{Cấu trúc đồ án}

Đồ án này gồm 6 chương, nội dung tương ứng của các chương như sau:

\begin{itemize}
    \item Chương 1: Giới thiệu.
          \begin{itemize}
              \item Tổng quan về SmartHome và nhu cầu co dãn tài nguyên.
              \item Đặt vấn đề, mô tả bài toán, mục tiêu và đóng góp của đồ án.
              \item Cấu trúc đồ án.
          \end{itemize}

    \item Chương 2: Tổng quan về công nghệ và cơ sở lý thuyết.
          \begin{itemize}
              \item Trình bày các công nghệ sử dụng trong đồ án như \gls{iot}, Apache Storm, Docker, giám sát hệ thống.
              \item Tổng quan về hệ thống xử lý dữ liệu.
              \item Điện toán đám mây công cộng.
              \item Công cụ quản lý cơ sở hạ tầng.
              \item Cơ sở lý thuyết của học tăng cường Q-learning.
          \end{itemize}

    \item Chương 3: Xây dựng giải pháp và thiết kế.
          \begin{itemize}
              \item Mô tả giải pháp.
              \item Thiết kế hệ thống co dãn tài nguyên dựa trên học tăng cường Q-learning.
              \item Xây dựng mô hình học tăng cường.
          \end{itemize}

    \item Chương 4: Triển khai hệ thống.
          \begin{itemize}
              \item Thiết lập mô hình giám sát, quản lý
              \item Xây dựng mô hình đánh giá hiệu quả hệ thống co/dãn.
              \item Triển khai hệ thống huấn luyện và thực nghiệm.
          \end{itemize}

    \item Chương 5: Huấn luyện, thử nghiệm và đánh giá.
          \begin{itemize}
              \item Thiết kế các kịch bản thử nghiệm.
              \item Đánh giá hiệu suất hệ thống.
              \item Đánh giá chi phí vận hành và khả năng tối ưu hóa.
              \item Đánh giá ưu điểm và nhược điểm của hệ thống.
              \item So sánh các thuật toán được sử dụng trong hệ thống.
              \item Phân tích khả năng mở rộng của hệ thống.
          \end{itemize}

    \item Chương 6: Kết luận và định hướng phát triển tiếp theo.
          \begin{itemize}
              \item Tổng kết các kết quả của đồ án.
              \item Thảo luận về các thách thức và hạn chế.
              \item Đề xuất các hướng nghiên cứu và phát triển trong tương lai.
          \end{itemize}
\end{itemize}