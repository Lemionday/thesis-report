\chapter{Mở đầu}

\section{Đặt vấn đề}

Trong bối cảnh hiện nay, khi tốc độ chuyển đổi số đang rất nhanh và không ngừng gia tăng, khi các thiết bị, vật dụng hằng ngày không ngừng được cải tiến, nâng cấp để nâng cao trải nghiệm người dùng, thì những công nghệ như đồng hồ thông minh, đèn điện tự động, khóa cửa từ đang dần trở thành một phần tất yếu trong không gian sống của mỗi các nhân.

Mạng lưới tập hợp các thiết bị thông minh và công nghệ tạo điều kiện thuận lợi cho hoạt động giao tiếp giữa thiết bị và đám mây cũng như giữa các thiết bị với nhau hình thành nên một mạng lưới mang tên Internet of Things (IoT) hay còn gọi là Internet vạn vật - thứ góp phần kiến tạo nên một môi trường sống thông minh, tiện nghi và linh hoạt chưa từng có.

Các thiết bị này ngày nay không còn chỉ thực hiện các chức năng đơn lẻ mà thông qua khả năng thu thập dữ liệu dựa trên các cảm biến về môi trường xung quanh, chúng đã có thể hỗ trợ cải thiện hệ thống một cách chủ động, hệ quả là nâng cao chất lượng trải nghiệm người dùng.

% tiếp

Tuy nhiên, khả năng thu thập dữ liệu này cũng có mặt hạn chế nhất định. Trong đó, có thể kể đến chính là lượng dữ liệu được gửi liên tục về hệ thống theo thời gian thực, đòi hỏi phải được xử lý với tốc độ nhanh và độ chính xác cao. Đây thực sự là bài toán khó đối với các hệ thống xử lý dữ liệu từ khâu thiết kế, triển khai cho đến vận hành.

Nhiều hệ thống xử lý luồng dữ liệu thời gian thực, điển hình là Apache Storm vốn không hề được trang bị cơ chế tài nguyên linh hoạt. Điều này khiến cho việc quản lý và duy trì vận hành hệ thống một cách thủ công trở nên tốn kém, thiếu hiệu quả, và dễ xảy ra sai sót.

Từ đây, vấn đề được đặt ra là: Làm sao để hệ thống có khả năng co dãn linh hoạt, chủ động - vừa tối ưu tài nguyên, vừa đảm bảo hiệu năng ổn định - không chỉ ở thời điểm hiện tại mà còn thích ứng tốt trong tương lai khi cả quy mô hệ thống lẫn khối lượng dữ liệu đều sẽ tăng theo thời gian?

Trên cơ sở đó, nghiên cứu này đề xuất giải pháp về co giãn tài nguyên đa cấp độ theo hướng chủ động, đồng thời vẫn tận dụng các ưu điểm vượt trội về khả năng xử lý thời gian thực, khả năng chịu lỗi và đảm bảo dữ liệu của hệ thống SmartHome đã được triển khai trên nền tảng Apache Storm.